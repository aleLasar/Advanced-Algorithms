\chapter{Introduzione\label{sec:introduzione}}
\noindent La relazione descrive gli algoritmi implementati da Alessio Lazzaron e Matteo Marchiori per il primo laboratorio del corso di algoritmi avanzati.

\section{Descrizione degli algoritmi\label{sec:descrizione}}
Gli algoritmi implementati sono:
\begin{itemize}
    \item Prim, implementato con Heap;
    \item Kruskal nella sua implementazione naive di complessità O(mn);
    \item Kruskal implementato con Union-Find
\end{itemize}

In seguito riportiamo lo pseudocodice degli algoritmi, e la parte saliente del codice implementato in Python in modo da confrontarli facilmente.
Per rendere più facile la comprensione della relazione, indichiamo con Kruskal-naive l'implementazione dell'algoritmo di Kruskal con complessità O(mn) mentre ci riferiamo semplicemente con Kruskal al medesimo algoritmo che utilizza la struttura dati Union-Find.
\clearpage

\subsection{Algoritmo di Prim\label{sec:prim}}
\begin{verbatim}
    Prim(G,s)
        foreach u in V:
            key[u] = infinity
            parent[u] = null
        key[s] = 0
        Q = V
        while Q not empty:
            u = extractMin(Q)
            foreach v adjacent_to u:
                if v in Q and w(u,v) < key[v]:
                    parent[v] = u
                    key[v] = w(u,v)
        return V\Q
\end{verbatim}

\begin{minted}{python}
    def prim(graph, s):
        mst_weight = 0
        adjacency_list = graph.get_graph()
        adjacency_list[s].set_key(0)
        heap_keys = Heap()
        for node in adjacency_list:
            heap_keys.push(node)
        while len(heap_keys) !=0:
            u = heap_keys.pop()
            adjacents = adjacency_list[u.name()].edges()
            for edge in adjacents:
                v = edge.node()
                if( v.present() and edge.weight() < v.key()):
                    v.set_parent(u)
                    v.set_key(edge.weight())
                    heap_keys._orderup(v._position)
            mst_weight += u.key()
        return mst_weight
\end{minted}

Abbiamo implementato la struttura Heap rappresentandola come una lista di nodi del grafo, che possiedono una chiave per cui vale la proprietà di min heap.
I nodi possiedono inoltre un campo che identifica la posizione all'interno dello heap, per poterlo riordinare quando vengono spostati, e un campo present per verificare se il nodo in questione fa parte dello heap o meno.
Per rappresentare il grafo abbiamo usato una lista di adiacenza, in cui ogni nodo ha una lista di archi incidenti in esso, e ogni arco mantiene l'informazione relativa al nodo opposto.

\clearpage

\subsection{Algoritmo di Kruskal-naive}
\begin{verbatim}
    Kruskal(G)
        A = {}
        sort edges of G by cost (mergesort)
        for each edge e in nondecreasing order of cost do:
            if A U {e} is acyclic then:
                A = A U {e}
        return A
\end{verbatim}    

\begin{minted}{python}
    def kruskal(graph, s):
    graph.ordinaLati()
    mst_weight = 0
    visitati = [False] * graph.num_vertici()
    componente = 0
    for edge in graph.get_edges(): 
        node1 = edge.nodes()[0]
        node2 = edge.nodes()[1]
        edge.set_mst(True)
        if node1 == node2:
            edge.set_mst(False)
        elif not visitati[node1.name()] and not visitati[node2.name()]:
            componente+=1
            node1.set_component(componente)
            node2.set_component(componente)
        elif not visitati[node1.name()]:
            node1.set_component(node2.component())
        elif not visitati[node2.name()]:
            node2.set_component(node1.component())
        else:
            if node1.component() != node2.component():
                fix_component(graph, node1)
            else:
                edge.set_mst(False)
        visitati[node1.name()] = True
        visitati[node2.name()] = True        
        if edge.mst():
             mst_weight += edge.weight()          
    return mst_weight
\end{minted}

L'algoritmo prevede l'ordinamento dei lati in modo crescente.
Per realizzarlo in modo efficiente abbiamo implementato un metodo nella classe Graph che utilizza l'algoritmo Merge Sort per ordinarli.
In questa versione dell'algoritmo verifichiamo su quali componenti connesse incide un lato del minimum spanning tree in costruzione.
Nel caso in cui il lato non incida in nessuna componente connessa, quest'ultimo andrà a creare una nuova componente connessa.
Invece, se il lato che si cerca di aggiungere va ad incidere su due nodi che appartengono alla medesima componente connessa allora si formerebbe un ciclo e quindi il lato non viene aggiunto.
Mentre se vengano connesse due componenti connesse distinte viene usato un approccio DFS per ridurle ad un'unica componente connessa.
In tutti gli altri casi il lato viene semplicemente aggiunto.

\clearpage

\subsection{Algoritmo di Kruskal\label{sec:kruskal}}
\begin{verbatim}
    Kruskal(G)
        A = {}
        U = initialize(V)
        sort edges of E by cost
        for each edge = (v,w) in non decreasing order of cost do:
            if find(U,v) != find(U,w):
                A = A U {(v,w)}
                Union(U,v,w)
        return A
\end{verbatim}

\begin{minted}{python}
    def kruskal(graph, s):
    mst_weight = 0
    mst = Graph(graph.num_vertici())
    union_find = UnionFind(graph.num_vertici())
    graph.ordinaLati()
    for edge in graph.get_edges():
        x = union_find.find(edge.nodes()[0].name())
        y = union_find.find(edge.nodes()[1].name())
        if x[0] != y[0]:
            s1 = mst.add_node(edge.nodes()[0].name())
            d1 = mst.add_node(edge.nodes()[1].name())
            mst.add_edge(s1, d1, edge._weight)
            union_find.union(s1.name(), d1.name())
            mst_weight += edge.weight()    
    return mst_weight
\end{minted}

Per rappresentare la struttura Union-Find abbiamo usato una lista di interi, che rappresentano i nomi dei nodi.
La nostra implementazione, di tale struttura, rispetta quanto visto a lezione sia per complessità, sia per le operazioni permesse (init, union e find).

\subsection{Scelte implementative comuni\label{sec:comuni}}
Per tutti gli algoritmi implementati abbiamo cercato di usare, per quanto possibile, la definizione di oggetti e la definizione di metodi previsti da python, in modo da mantenere per quanto possibile il codice comprensibile.