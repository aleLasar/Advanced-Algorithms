\chapter{Introduzione\label{sec:introduzione}}
\noindent La relazione descrive gli algoritmi implementati da Alessio Lazzaron e Matteo Marchiori per il secondo laboratorio del corso di algoritmi avanzati.

\section{Descrizione degli algoritmi\label{sec:descrizione}}
Gli algoritmi implementati sono:
\begin{itemize}
    \item Held Karp;
    \item Algoritmo 2-approssimato, con MST costruito usando l'algoritmo di Prim;
    \item Euristica costruttiva Cheapest Insertion.
\end{itemize}

In seguito riportiamo lo pseudocodice degli algoritmi, e la parte saliente del codice implementato in Python in modo da confrontarli facilmente.

\subsection{Algoritmo di Held Karp\label{sec:hp}}
\begin{verbatim}
    Held-Karp(G=(V,E), S, v):
        if(S = {v}) return w[v,0]
        if(d[v,S] is not NULL) return d[v,S]
        mindist = inf
        minprec = NULL
        for u in S\{v} do:
            dist = Held-Karp(u, S\{v})
            if dist + w[u,v] < mindist:
                mindist = dist + w[u,v]
                minprec = u
        d[s,V] = mindist
        p[s,V] = minprec
        return mindist        
\end{verbatim}

\clearpage

\begin{minted}{python}
    def held_karp(start, v, S: list, d_dict, p_dict):
        global stop
        if len(S) == 1 and v in S:
            return v.find_edge(start).weight()
        elif (str(v.name()+1), str(S)) in d_dict:
            return d_dict[(str(v.name()+1), str(S))]
        else:
            mindist = float("Inf")
            minprec = None
            S2 = list(S)
            S2.remove(v)
            for i in range(len(S2)):
                dist = held_karp(start, S2[i], S2, d_dict, p_dict)
                uv_weight = S2[i].find_edge(v).weight()
                if(dist + uv_weight < mindist):
                    mindist = dist + uv_weight
                    minprec = S2[i]
                if stop:
                    break
            d_dict[(str(v.name()+1), str(S))] = mindist
            p_dict[(str(v.name()+1), str(S))] = minprec
            return mindist
\end{minted}

Per tenere memoria delle distanze minime e dei nodi predecessori abbiamo utilizzato dizionari di python, in modo da mantenere l'indicizzazione attraverso (nodo destinazione , lista dei nodi coperti) vista a lezione.
Avendo l'algoritmo una complessità O(\(2^{n}\)) è stata inserito un meccanismo per fermare il calcolo dopo un numero di secondi specificabile non noto a priori.

\clearpage

\subsection{Algoritmo 2-approssimato\label{sec:a2a}}
\begin{verbatim}
    Preorder(v):
        print(v)
        if(internal(v)):
            foreach u in children(v) do:
                preorder(u)

    Approx_t_tsp(G=(V,E), c):
        V = {v1...vn}
        root = v1
        T = Prim(G,c,root)
        <v1...vn> = Preorder(root)
        return <v1...vn, v1>            
\end{verbatim}

\begin{minted}{python}
    def preorder(preordered, v):
        preordered.append(v)
        for child in v.children():
            preorder(preordered, child)


    def approx_t_tsp(graph, s):
        prim(graph, s)
        nodes = graph.get_graph()
        preordered = []
        preorder(preordered, nodes[s])
        preordered.append(nodes[s])
        return preordered
\end{minted}

In questo algoritmo costruiamo una lista dei nodi di un MST trovato usando Prim in ordine prefisso, aggiungendo al termine della lista il nodo radice.
Funziona perché vale la disuguaglianza triangolare, ovvero per arrivare da un nodo sorgente a un nodo destinazione la soluzione migliore include l'arco che li collega direttamente.
Osservando la soluzione peggiore di TSP trovata con l'algoritmo, si vede che non può essere peggiore rispetto a prendere due volte il costo della soluzione ottima.
Quindi l'algoritmo di 2-approssimazione è corretto.

\clearpage

\subsection{Algoritmo Cheapest Insertion\label{sec:ci}}
\begin{verbatim}
    Cheapest_insertion(G=(V,E)):
        sol = (0,j,0)
        do:
            (k,i)+(k,j)-(i,j) minimize cost
            sol = sol + (k,i) + (k,j) - (i,j)
        while(exists k outside sol)    

\end{verbatim}

\begin{minted}{python}
    def cheapest_insertion(G:Graph):
    edges = G.edges()
    nodes = list(G.nodes())
    edges_sol = {}
    weight_sol = 0
    zero_n = nodes.pop(0)
    
    while len(nodes) != 0:
        local_min = float("inf")
        local_edge = [None]*3
        local_node = None
        if len(edges_sol) == 0:
            for i in range(len(nodes)):
                val = edges[zero_n.name(), nodes[i].name()]
                if val.weight() < local_min:
                    local_min = val.weight()
                    local_node = nodes[i]
                    local_edge[0] = val
        else:
            for k in range(len(nodes)):
                for idx, val in enumerate(edges_sol.values()):
                    ik = edges[nodes[k].name(), val.nodes()[0].name()]
                    jk = edges[nodes[k].name(), val.nodes()[1].name()]
                    ij = val

                    to_minimized = ik.weight() + jk.weight() - ij.weight()
                    if to_minimized < local_min:
                        local_min = to_minimized
                        local_node = nodes[k]
                        local_edge = ik, jk, val

        nodes.remove(local_node)
        if not local_edge[1]:
            weight_sol += local_min*2
            nodes_e = local_edge[0].nodes()
            edges_sol[nodes_e[0].name(), nodes_e[1].name()] = local_edge[0]

            
        else:
            weight_sol += local_edge[0].weight() + local_edge[1].weight()\
            - local_edge[2].weight()
            nodes_0 = local_edge[0].nodes()
            nodes_1 = local_edge[1].nodes()
            nodes_2 = local_edge[2].nodes()
            edges_sol[nodes_0[0].name(), nodes_0[1].name()] = local_edge[0]
            edges_sol[nodes_1[0].name(), nodes_1[1].name()] = local_edge[1]
            del edges_sol[nodes_2[0].name(), nodes_2[1].name()]

    return weight_sol
\end{minted}

Questo algoritmo di euristica costruttiva parte da un nodo radice, sceglie l'arco di costo minimo e il nodo destinazione.
Dopodiché viene preso ogni volta un nodo non presente nella soluzione in modo da minimizzare il costo complessivo della soluzione trovata, ovvero prendendo quel nodo non incluso nella soluzione tale che la somma del peso degli archi per connetterlo togliendo l'arco che connetteva i due nodi a cui quest'ultimo è connesso sia la più piccola possibile.

\subsection{Scelte implementative comuni\label{sec:comuni}}
Per tutti gli algoritmi implementati abbiamo cercato di usare, per quanto possibile, la definizione di oggetti e la definizione di metodi previsti da python, in modo da mantenere per quanto possibile il codice comprensibile.