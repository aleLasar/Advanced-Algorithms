\chapter{Originalità\label{sec:originalita}}
\noindent In questa sezione vengono presentate le originalità che abbiamo inserito nella nostra implementazione dei tre algoritmi.

\section{Kruskal naive\label{sec:originalitanaives}}

Per quanto riguarda l'implementazione di Kruskal-naive, per cercare la ciclicità ci siamo basati sulle componenti connesse nella costruzione di un MST partendo da un grafo, principio simile a quello adottato da Kruskal con Union Find.
Quando viene aggiunto un lato al MST, si controllano i nodi su cui tale lato incide.
Se tali nodi fanno parte della stessa componente connessa, allora il lato non viene aggiunto perché altrimenti formerebbe un ciclo.
In caso contrario viene aggiunto, e si aggiornano le componenti connesse di alcuni nodi, in modo da ottenerne una sola.

Per come l'abbiamo implementata, la ricerca di ciclicità è corretta per costruzione.

Inoltre sembra che la complessità sia inferiore a O(mn), perché qualora i nodi appartengano a due componenti distinte e venga aggiunto un arco tra essi, vengono aggiornati solamente i nodi del MST con componente connessa da aggiornare.
Resta in ogni caso superiore a Prim e Kruskal, perché non viene fatta alcuna considerazione sul modo in cui viene aggiornata tale componente.
Nel caso peggiore, ovvero quando il grafo dato fosse un MST, e gli archi avessero una disposizione tale da essere aggiunti in modo da aggiornare sempre la componente connessa di n-1 nodi, la complessità risulterebbe O(mn/2), quindi O(mn).