\chapter{Introduzione\label{sec:introduzione}}
\noindent La relazione descrive l'algoritmo implementato da Alessio Lazzaron e Matteo Marchiori per il terzo laboratorio del corso di algoritmi avanzati.

\section{Descrizione dell'algoritmo\label{sec:descrizione}}
L'unico algoritmo implementato per questo laboratorio è quello di Karger.
In seguito poniamo lo pseudocodice e la parte interessante del codice in python per un rapido confronto, e alcune scelte fatte durante l'implementazione.

\subsection{Algoritmo di Karger\label{sec:karger}}
\begin{verbatim}
    Full_contraction(G=(V,E)):
        for i = 1 to |V| - 2 do:
            e = random(E)
            G' = (V, E') = G/e
            V = V'
            E = E'
        return |E|    
    
    Karger(G=(V,E), k):
        min = infinity
        for i = 1 to k do:
            t = Full_contraction(G)
            if t < min:
                min = t
        return min                
\end{verbatim}

\clearpage

\begin{minted}{python}
    def full_contraction(graph):
        nodes = graph.nodes()
        not_removed = list(nodes)
        for i in range(len(nodes)-2):
            src_index = random.randint(0, len(not_removed)-1)
            src = not_removed[src_index]
            adjacents_src = src.adjacents()
            dest = random.choice(adjacents_src)
            dest_index = dest.index()
            adjacents_dest = dest.adjacents()
            newname = graph.inc_lastnode()
            adjacents_src = [
                adjacent for adjacent in adjacents_src if adjacent != dest
            ]
            adjacents_dest = [
                adjacent for adjacent in adjacents_dest if adjacent != src
            ]
            newadjacents = adjacents_src + adjacents_dest
            newnode = Node(newname, dest_index)
            newnode.set_adjacents(newadjacents)
            for node in newadjacents:
                adjacents = node.adjacents()
                for i, adjacent in enumerate(adjacents):
                    if adjacent is src or adjacent is dest:
                        adjacents[i] = newnode
            nodes[src_index] = None
            nodes[dest_index] = newnode
            del not_removed[src_index]

        mincut = 0
        for node in not_removed:
            if node is not None:
                for adjacent in node.adjacents():
                    if adjacent is not None:
                        mincut += 1
                break
        return mincut
\end{minted}        

\clearpage

\begin{minted}{python}
    def karger(graph, k, ottimo):
        mincut = float("inf")
        original = copy.deepcopy(graph)
        time_fc = 0
        time_discovery = 0
        found = False
        for _ in range(k):
            fc_start = time.time()
            fc = full_contraction(graph)
            if fc == ottimo and not found:
                time_discovery = time.time()
                found = True
            elif fc < ottimo:
                time_discovery = time.time()
                ottimo = fc
            time_fc += time.time() - fc_start
            if fc < mincut:
                mincut = fc
            graph = copy.deepcopy(original)
        time_fc = time_fc / k
        return mincut, time_fc, time_discovery

\end{minted}

Nell'algoritmo \textit{full\_contraction} la contrazione di due nodi avviene creando un nuovo nodo nel grafo che unisce due nodi presi randomicamente: il primo dai nodi non ancora eliminati mentre il secondo tra i possibili nodi adiacenti al primo, in modo da avere almeno un arco esistente tra i due.
Inoltre il nuovo nodo avrà come nodi adiacenti i nodi adiacenti dei nodi scelti e verrà memorizzato al posto del secondo nodo, sovrascrivendolo, mentre il primo verrà eliminato.
Alla fine la procedura restituisce il taglio trovato.

La funzione chiamata \textit{Karger}, invece, fa eseguire la procedura \textit{full\_contraction} k volte, dove k viene determinato con la seguente formula: 

\centerline{\(\frac{n^{2}}{2}\cdot ln(n)\)}

Inoltre prima di ogni nuova chiamata alla funzione \textit{full\_contraction} il grafo viene impostato alla versione fornita in input e viene calcolato il tempo di esecuzione. Tuttavia nella stima dei tempi dell'algoritmo di Karger è presente uno scarto temporale che dipende dalla dimensione del grafo e dal parametro k.

Per tutti gli algoritmi implementati abbiamo cercato di usare, per quanto possibile, la definizione di oggetti e la definizione di metodi previsti da python, in modo da mantenere per quanto possibile il codice comprensibile.