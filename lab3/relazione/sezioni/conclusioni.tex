\chapter{Conclusioni\label{sec:conclusioni}}

\section{Risultati ottenuti\label{sec:risultati-ottenuti}}
Dai risultati ottenuti è visibile come i tempi trovati rispecchino la complessità dell'algoritmo stimata.

Essendo il discovery time molto basso, specie nei grafi di dimensioni maggiori, abbiamo preferito riportarlo solamente in tabella (dal grafico non era esposto chiaramente).
Si spiega con il fatto che la probabilità di errore decresce con il numero di nodi presenti nel grafo, secondo la stima di k.

L'errore calcolato è sempre pari a 0, infatti la probabilità di errore si è dimostrata essere così bassa per ogni grafo da trovare sempre il risultato aspettato. Non è detto che questo avvenga sempre.

Per avere una stima più corretta dei tempi medi impiegati dall'algoritmo di Karger sarebbe necessario togliere il tempo impiegato per la copia del grafo, che potrebbe influire negativamente nelle istanze di dimensione maggiore.