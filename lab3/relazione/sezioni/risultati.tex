\chapter{Risultati\label{sec:risultati}}
\noindent In questa sezione mostriamo i risultati ottenuti dall'algoritmo, ovvero i tempi medi impiegati per full contraction e karger, il discovery time, la soluzione trovata, la soluzione attesa e l'errore relativo.

\section{Grafici\label{sec:grafici}}

\subsection{Full contraction\label{sec:fc}}

Il presente grafico illustra i tempi di calcolo della procedura Full\_Contraction sui grafi del dataset al variare del numero di vertici. Confrontiamo con la complessità asintotica stimata di \(O(n^{2})\).

\subsection{Karger\label{sec:karger}}

Il presente grafico illustra i tempi di calcolo dell'algoritmo di Karger sui grafi del dataset al variare del numero di vertici. La probabilità di errore è uguale a \(\frac{1}{n}\) di sbagliare per grafi inferiori a $75$ nodi. Tale scelta è dovuta a motivi di tempo, come illustrato in sezione \vref{sec:originalita}. Confrontiamo con la complessità asintotica stimata di \(O(n^{4}log(n))\).

\subsection{Discovery time\label{sec:dt}}

Il presente grafico illustra i tempi di calcolo dell'algoritmo di Karger per trovare la soluzione ottima per la prima volta. Confrontiamo i risultati con il tempo di esecuzione complessivo.


\section{Tabella\label{sec:tabella}}

\footnotesize
\begin{tabularx}{\textwidth}{*{10}{Y}}
    \toprule
    \textbf{Istanza} & \textbf{Full contraction} & \textbf{Karger} & \textbf{Discovery time} & \textbf{Risultato} & \textbf{Ottimo} & \textbf{Errore relativo}\\
    \endfirsthead
    \toprule
    \textbf{Istanza} & \textbf{Full contraction} & \textbf{Karger} & \textbf{Discovery time} & \textbf{Risultato} & \textbf{Ottimo} & \textbf{Errore relativo}\\
    \endhead
    \midrule
    
    \bottomrule
    \caption{Risultati}\label{tab:risultati}
\end{tabularx}

\normalsize

\clearpage

\section{Risultati\label{sec:risultati}}
Dai risultati ottenuti si può vedere che malgrado l'algoritmo di Held Karp sia un algoritmo esatto richiede molto tempo per trovare il risultato cercato. Troncando il tempo di esecuzione a tre minuti, la soluzione esatta viene trovata solamente per i grafi più piccoli, composti al massimo da $16$ nodi.
Gli algoritmi cheapest insertion e 2-approssimazione invece impiegano un tempo accettabile su tutti i grafi (entro i $10$ minuti impiegati da cheapest insertion per il grafo più grande). Questo è dovuto alla complessità molto inferiore rispetto a quella di Held Karp, che è esponenziale.
L'algoritmo migliore, rispetto all'errore di approssimazione, risulta essere stranamente cheapest insertion, pur non promettendo un bound superiore garantito.
Da quanto visto a lezione tale limite dovrebbe essere uguale a quello dell'algoritmo di 2-approssimazione, che effettivamente risulta rispettato.
L'algoritmo più veloce nel calcolo dell'approssimazione è quello di 2-approssimazione, che commette però un errore più elevato rispetto agli altri due algoritmi (in realtà l'errore più elevato è commesso da Held Karp con interruzione, ma senza considerare un'interruzione ai tre minuti tale algoritmo fornirebbe un risultato esatto).